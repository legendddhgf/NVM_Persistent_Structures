\documentclass{article}

\usepackage[utf8]{inputenc}
\usepackage{lipsum}                     % Dummytext
\usepackage{hyperref}
\usepackage{xargs}                      % Use more than one optional parameter in a new commands
\usepackage[pdftex,dvipsnames]{xcolor}  % Coloured text etc.
\usepackage{graphicx}
\usepackage{verbatim}
\usepackage{float}
\usepackage{tikz-qtree}
\usepackage{tikz}
\usepackage[linguistics]{forest}

\usepackage{amssymb}
\usepackage{amsmath}
\newcommand*{\QEDA}{\hfill\ensuremath{\blacksquare}}% filled box
\newcommand*{\QEDB}{\hfill\ensuremath{\square}}% unfilled box

% dem nice tables
\usepackage[hmargin=2cm,top=4cm,headheight=65pt,footskip=65pt]{geometry}
\usepackage{fmtcount} % for \ordinalnum
\usepackage{array,multirow}
\usepackage{tabularx}
\usepackage{lastpage}


% add a special collumn type
\newcolumntype{C}[1]{>{\centering\arraybackslash}m{#1}}


%header/footer stuff
\usepackage{fancyhdr}
\pagestyle{fancy}

%note that if you do not do these blank ones, the package defaults to something
%you may not want in your header or footer
\lhead{CMPE 293}
\chead{}
\rhead{\today}
\lfoot{Isaak Cherdak}
\cfoot{}
\rfoot{\thepage}

\renewcommand{\headrulewidth}{0pt}
\renewcommand{\footrulewidth}{0pt}

\hypersetup{
    colorlinks=true,
    linkcolor=blue,
    filecolor=magenta,
    urlcolor=cyan,
}

\usepackage[english]{babel}
\emergencystretch=1pt
\usepackage[justification=centering]{caption}
\graphicspath{{Pictures/} }

\usepackage[colorinlistoftodos,prependcaption,textsize=tiny]{todonotes}
\newcommandx{\unsure}[2][1=]{\todo[linecolor=red,backgroundcolor=red!25,bordercolor=red,#1]{#2}}
\newcommandx{\change}[2][1=]{\todo[linecolor=blue,backgroundcolor=blue!25,bordercolor=blue,#1]{#2}}
\newcommandx{\info}[2][1=]{\todo[linecolor=OliveGreen,backgroundcolor=OliveGreen!25,bordercolor=OliveGreen,#1]{#2}}
\newcommandx{\improvement}[2][1=]{\todo[linecolor=Plum,backgroundcolor=Plum!25,bordercolor=Plum,#1]{#2}}
\newcommandx{\thiswillnotshow}[2][1=]{\todo[disable,#1]{#2}}

\usepackage{setspace}
\doublespacing

\title{A Comparison of Persistent Data Structures for Non-Volatile Memory
Applications}
\author{Isaak Cherdak}
%\date{} %blank date

\begin{document}

\maketitle

\section*{Abstract}

With the introduction of byte-addressable persistent memory comes the need to
rethink the way that data in memory is interfaced and structured. At the heart
of this research is the consideration of various persistent data structures.
This paper seeks to determine the benefits and trade-offs of four different
persistent data structures and suggest scenarios in which each one is best
utilized.

\pagebreak

\tableofcontents

\pagebreak

\todo[inline]{setup bibtek crap}

\section{Introduction}

Non-volatile memory (NVM) systems research has become a major focus. NVM
provides the possibility of byte-addressable memory that persists across power
cycles while maintaining latencies similar to that of DRAM. Although NVM
technologies have yet to reach their intended potential, there are a number of
topics of theoretical research that can be done so that NVM can be fully
utilized once it is more accessible. For example, with volatile memory, any
corruption or bugs present in memory can be cleared with a power cycle but with
NVM this is no longer an option. Thus, one requirement of structuring data in
NVM is the use of persistent data structures, or data structures whose state
cannot be invalidated. However, there are many data structures to choose from
and the potential for new ones to be created with particular design focus (i.e.
batch read optimized).

This paper will compare four different persistent data structures in various
scenarios. First some background on major non-volatile memory system research
concerns will be discussed. Then the various scenarios used to compare data
structures in this paper are described including their uses in both consumer and
business environments. Next the implementation of the data structures is
discussed including informal proof of their persistent correctness. Finally the
paper discusses results of the comparisons and concludes with suggestions for
the structures and a general statement about what these results mean about
certain properties of certain kinds of data structures.

\section{Background}

\section{Methodology}

My focus was on ...

\section{Implementation}

\subsection{Overview}

We focused on making things persistent (invalid state not possible). We also
used clwb
(\url{https://software.intel.com/sites/landingpage/IntrinsicsGuide/#text=clwb&expand=662}
from intrinsics refernce).

We multiplied results from cachegrind to reflect NVM timings accordingly...

We further used pilot to aid in analysis of the results and structuring the
benchmark.

\section{Results}

\section{Conclusion}

\section{Citation Info}

\todo[inline]{This section is temporary and should be made using bibtek later}

sources to consider:

B-Trees:\\
\url{https://dl.acm.org/citation.cfm?doid=356770.356776}

\end{document}
